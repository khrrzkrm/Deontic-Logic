\title{Review of section 11: Intermediate concepts}
\author{Katarzyna Frackowski, Falko Hemstra}
\date{\today}
  
\documentclass[12pt]{article}

\usepackage{csquotes}
\usepackage{amssymb}
\usepackage{hyperref}
\usepackage{color}
\usepackage{soul}

\begin{document}
\maketitle



Chapter 11 of Pigozzi’s and van der Torre’s paper \cite{pigozzi2017multiagent} describes how deontic logic semantics should be enhanced with meaning postulates and intermediate concepts. They commence with defining the terms of meaning postulates and intermediate concepts, to immediately resolve that these are so far not embodied by deontic logic. They conclude with a proposal on how to extend deontic logic semantics to incorporate the helpful explanatory concepts.

In order to understand the importance of those concepts it is necessary to decode their order of precedence in legal contexts. On a very high level, a complex legal term consists of a legal definition which can also be described as a meaning postulate. This normally comprises more basic legal terms whose legal definitions link them to words describing natural facts, and this relation can be considered an intermediate concept.

\emph{Meaning postulates} which are defined as a way to stipulate a relationship between two or more words, are different from norms in the way that they do not \enquote{prescribe or allow a behavior} and are held true by definition. It is important to understand that meaning postulates consist of multiple more basic concepts that identify the stipulative definition. These are known as \emph{intermediate concepts} that \enquote{link legal terms [...] to words describing natural facts.}

Pigozzi and van der Torre state that current concepts of deontic logic are not yet developed well enough to comprise the above mentioned legit ideas which explain and decompose legal norms into the most basic coherent elements and how they relate.

Since intermediate concepts would improve the understanding of contractual implications of legal norms as well as reduce the amount of necessary explanations and definitions, Pigozzi and van der Torre then try to make a proposal on how they should be included in deontic logic. 
The first idea is to set up relational terms. The authors describe their idea with the example of the term  \enquote{$\mathit{Ownership}(x,y)$} that shall imply the relation between the terms $F_1,...,F_p$, which are the facts describing a change of ownership of objects,  and $C_1,...,C_n$, describing the legal consequences of that very change. This would reduce the number of potential implications from a $p \times n$ multiplication to a $p + n$ addition: p implications will link all the facts to the new term \enquote{ownership}, and n implications will link all the consequences to the \enquote{ownership}. This does not hold for situations where different facts would have the same consequence and hence, one implication will suffice to represent the ownership term. But if we consider that these facts and consequences come from different normative systems, it is understandable that they cannot be generalized in the above mentioned way.

The second idea refers to input/output logic. Meaning postulates should then be defined as intermediate concepts which consist of a fact and a consequence that --- treated as an input --- will result in an output described along a valid set of norms. According to Pigozzi and van der Torre such a setup would also identify poorly defined contracts with regard to the applied terms. Giving an example of forbidding dogs on premises but exempting blind persons, they illustrate that common understanding would naturally resolve the described conflict, but without a natural comprehension such an intermediate would require a more precise definition.

Pigozzi and van der Torre make their point in emphasizing the complexity of intermediate concepts and meaning postulates for deontic logic, and how difficult it is to resolve this complexity in a logic framework. They point out the risks associated with misuse and/or imprecise logical detail work. For their purposes, they mostly explain high level concepts and examples, and fall a little short on providing a satisfactory abstraction and formalization of the required logical enhancement. Given the complexity of this topic though, it is obvious that the concepts require a much broader analysis and conception, which cannot be explained well in just two pages.

\bibliographystyle{abbrv}
\bibliography{paper}

\end{document}
