% !TeX root = ../../presentation-local.tex

\section{Propositions as Types}

\begin{frame}{The Idea, 1}
\begin{itemize}
  \item We have seen a fairly standard functional programming language (with restricted recursion)

  \pause

  \item But wasn't Coq about giving proofs of propositions about programs?

  \pause

  \item \textbf{Roadmap}: We add very few features to the language and show how we can prove propositions \textbf{within} the language, as opposed to using some meta framework

  \pause

  \item We need to establish the following mechanisms:
  \begin{itemize}
    \item A \textbf{proposition} (logical statement) is encoded as a \textbf{type}
    \item A \textbf{proof} of a proposition is encoded as a \textbf{term} of that type
  \end{itemize}

  \pause

  \item The idea of using propositions as types is also called \textit{Curry-Howard-Correspondence}
\end{itemize}
\end{frame}

\begin{frame}{The Idea, 2}
  \textit{Spoilers}:\\[1em]
  \begin{minipage}[t]{0.42\textwidth}
       Implication $P \rightarrow Q$\\
       Conjunction $P ∧ Q$\\
       Disjunction $P ∨ Q$\\
       Top $⊤$\\
       Bottom $⊥$\\[1em]
       Univ. Qu. $∀(x:A), P(x)$\\
       Exis. Qu. $∃(x:A), P(x)$\\[1em]
       Modus Ponens:\\From $P \rightarrow Q$ and $P$ one deduces $Q$\\[1em]
       There is a proof tree for $P$
  \end{minipage}
  \begin{minipage}[t]{0.04\textwidth}
       $\tilde{=}$\\
       $\tilde{=}$\\
       $\tilde{=}$\\
       $\tilde{=}$\\
       $\tilde{=}$\\[1em]
       $\tilde{=}$\\
       $\tilde{=}$\\[1em]
       $\tilde{=}$\\[1em]
       ~\\~\\
       $\tilde{=}$\\
  \end{minipage}
  \begin{minipage}[t]{0.35\textwidth}
       Funct. type \lstinline|P -> Q|\\
       Product type \lstinline[mathescape=true]|P| $\times$ \lstinline|Q|\\
       Sum type \lstinline|P + Q|\\
       Unit type \lstinline|1|\\
       Empty type \lstinline|0|\\[1em]
       $Π$-types\\
       $\Sigma$-types\\[1em]
       Function application: \lstinline|f: P -> Q| and \lstinline|p: P| gives \lstinline|f p: Q|\\[1em]
       There is a term \lstinline|t| such that \lstinline|t: P|
  \end{minipage}
\end{frame}

\begin{frame}[fragile]{Top and Bottom}
\begin{itemize}
  \item Let's define the proposition $⊤$ (\say{truth})
  \item There should be a proof for $⊤$
\lstinputlisting[language=coqorig]{content/theorem-proving/coq/snippets/true.v}

  \pause

  \item Let's define the proposition $⊥$ (\say{falsity})
  \item There should be \textbf{no} proof for $⊥$
\lstinputlisting[language=coqorig]{content/theorem-proving/coq/snippets/false.v}
  ~\\[1em]

  \pause

  \item In Coq, there is a special type universe for propositions, called \lstinline|Prop| (more about universes later)
  \item From now on, we write \lstinline|True| for \lstinline[language=coqorig]|True| and \lstinline|False| for \lstinline[language=coqorig]|False|
\end{itemize}
\end{frame}

\begin{frame}[fragile]{Conjunction}
\begin{itemize}
  \item We now come to our first connective, conjunction
\lstinputlisting{content/theorem-proving/coq/snippets/and.v}

  \pause

  \item \lstinline|and| is a type constructor (better: \lstinline|Prop| constructor):
    \begin{itemize}
      \item Given two \lstinline|Prop|'s, it establishes a new \lstinline|Prop|
      \item The two \lstinline|Prop|'s are parameters (implicit for \lstinline|conj|)
      \item There is notation \lstinline|A /\  B| for \lstinline|and A B|
    \end{itemize}

  \pause

  \item How do you prove \lstinline|A /\  B|?

  \pause

  \item Give proofs \lstinline|a: A| and \lstinline|b: B| and apply \lstinline|conj| to them

  \pause

  \item Example: Prove \lstinline|True /\ True|
\lstinputlisting{content/theorem-proving/coq/snippets/t_and_t.v}
\end{itemize}
\end{frame}

\begin{frame}[fragile]{Disjunction}
\begin{itemize}
  \item We now come to our second connective, disjunction
\lstinputlisting{content/theorem-proving/coq/snippets/or.v}

  \pause

  \item \lstinline|or| is a type constructor (better: \lstinline|Prop| constructor):
    \begin{itemize}
      \item Given two \lstinline|Prop|'s, it establishes a new \lstinline|Prop|
      \item The two \lstinline|Prop|'s are parameters (implicit for \lstinline|or_introl|, \lstinline|or_intror|)
      \item There is notation \lstinline|A \/  B| for \lstinline|or A B|
    \end{itemize}

  \pause

  \item How do you prove \lstinline|A \/  B|?

  \pause

  \item Prove \lstinline|a: A| and apply \lstinline|or_introl| \textit{or}\\
  prove \lstinline|b: B| and apply \lstinline|or_intror|

  \pause

  \item Example: Prove \lstinline|False \/ True|
\lstinputlisting{content/theorem-proving/coq/snippets/f_or_t.v}
\end{itemize}
\end{frame}

\begin{frame}[fragile]{Types, so far}
    \only<1>{
\begin{itemize}
  \item Let's recall what we know about types in Coq so far
  \item First, how are types formed? We saw 3 possibilities:\\
\end{itemize}

  \begin{block}{1a) \lstinline|Inductive| types (atomic)}
    \begin{itemize}
      \item \lstinline|bool| \hfill \lstinline|: Type|
        \begin{itemize}
          \item with value \lstinline|true: bool|
        \end{itemize}
      \item \lstinline|False| \hfill \lstinline|: Prop|
        \begin{itemize}
          \item no proofs
        \end{itemize}
    \end{itemize}
  \end{block}
  \begin{block}{1b) \lstinline|Inductive| types (applied type constructors)}
    \begin{itemize}
      \item \lstinline|list nat| \hfill \lstinline|: Type|
        \begin{itemize}
          \item with value \lstinline|1::2::3::nil|
        \end{itemize}
      \item \lstinline|True /\\ True| \hfill \lstinline|: Prop|
        \begin{itemize}
          \item with proof \lstinline|conj I I|
        \end{itemize}
    \end{itemize}
  \end{block}
  }

  \only<2>{\begin{block}{2) Function Types}
    \begin{itemize}
      \item \lstinline|bool -> bool| \hfill \lstinline|: Type|
        \begin{itemize}
          \item with value \lstinline|negb: bool -> bool|
          \item with value \lstinline|(fun x => x): bool -> bool|
        \end{itemize}
      \item \lstinline|nat -> nat -> nat| \hfill \lstinline|: Type|
        \begin{itemize}
          \item with value \lstinline|plus: nat -> nat -> nat|
        \end{itemize}
      \item \lstinline|False -> False| \hfill \lstinline|: Prop|
        \begin{itemize}
          \item with...  a proof? Yes: \lstinline|fun f => f|
        \end{itemize}
      \item \lstinline|True -> False| \hfill \lstinline|: Prop|
        \begin{itemize}
          \item with...  a proof? No.
        \end{itemize}
    \end{itemize}
  \end{block}}

  \only<3>{\begin{block}{3) Polymorphic Types}
    \begin{itemize}
      \item \lstinline|forall (X: Type), list X| \hfill \lstinline|: Type|
        \begin{itemize}
          \item with value \lstinline|nil|
        \end{itemize}
      \item \lstinline|forall (X: Type), X -> list X| \hfill \lstinline|: Type|
        \begin{itemize}
          \item with value \lstinline|fun (X: Type) (x: X) => x::x::x::nil|
        \end{itemize}
      \item \lstinline|forall (A B: Prop), A -> B -> A /\\|~\lstinline|B| \hfill \lstinline|: Prop|
        \begin{itemize}
          \item with proof \lstinline|conj|
        \end{itemize}
      \item \lstinline|forall (P: Prop), True \\/|~\lstinline|P| \hfill \lstinline|: Prop|
        \begin{itemize}
          \item with... a proof? Yes:\\
          \lstinline|fun (_: Prop) => or_introl I|
        \end{itemize}
    \end{itemize}
  \end{block}}

  \only<3>{
    \begin{itemize}
      \item \textbf{Polymorphic types} are \textbf{function types} that take types as arguments
      \item \textbf{Polymorphic values} are \textbf{functions} that take types as arguments
    \end{itemize}}
\end{frame}

\begin{frame}[fragile]{Polymorphic Propositions}
\begin{itemize}
  \item As we have seen, propositions can be polymorphic, too
  \item Example: \textit{For all propositions \lstinline|P|, we have \lstinline|True \\/ P|}

  \pause

  \item We formulate and prove this proposition as follows:
\lstinputlisting{content/theorem-proving/coq/snippets/t_or_p.v}

  \pause

  \item Short notation:
\lstinputlisting{content/theorem-proving/coq/snippets/t_or_p_short.v}

  \pause

  \item The type is \textit{polymorphic} in \lstinline|P|

  \pause

  \item The proof is \textit{polymorphic} in \lstinline|P|
\end{itemize}
\end{frame}

\begin{frame}[fragile]{Implication}
\begin{itemize}
  \item We now come to our third logical connective, implication

  \pause

  \item You have seen it, as it is already built-in: An \textit{implication} is a \textit{function type}!

  \pause

  \item Example: Prove that for all propositions \lstinline|P|, we have \lstinline|P -> True /\|~\lstinline|P|.
\lstinputlisting{content/theorem-proving/coq/snippets/impl.v}

  \pause

  \item Short notation:
\lstinputlisting{content/theorem-proving/coq/snippets/impl_short.v}
\end{itemize}
\end{frame}


\begin{frame}[fragile]{Proving with Tactics, 1}
\begin{itemize}
  \item A conjunction can be proven as follows
\lstinputlisting{content/theorem-proving/coq/snippets/t_and_t.v}

  \pause

  \item We call the value of a proposition a \textbf{proof term}

  \pause

  \item This proof term can equivalently be obtained via \textbf{tactics}

\lstinputlisting{content/theorem-proving/coq/snippets/t_and_t_tactics.v}
  \pause

  \item Tactics \textbf{generate proof terms}

  \pause

  \item Display proof term via ~~\lstinline|Print t_and_t.|
\end{itemize}
\end{frame}

\begin{frame}[fragile]{Proving with Tactics, 2}
\lstinputlisting{content/theorem-proving/coq/snippets/t_and_t_tactics.v}

\begin{itemize}
  \item Enables backwards-directed reasoning
  \pause
  \item The \textit{goal} (proof obligation) is simplified/divided/reduced to smaller \textit{subgoals}
  \pause
  \item \lstinline|apply| can be used to apply a value constructor \lstinline|conj|
  \pause
  \item If the value constructor expects further arguments, further subgoals are generated
  \pause
  \item This is the case in our example: We have two subgoals \lstinline|True| and \lstinline|True|

\end{itemize}
\end{frame}

\begin{frame}[fragile]{Tactic: \lstinline|exact|}
\lstinputlisting{content/theorem-proving/coq/snippets/t_and_t_exact.v}
\begin{itemize}
  \item By \lstinline|exact|, one can give an explicit proof term
  \pause
  \item In this example, we give the whole proof term just by a single \lstinline|exact|, which is equivalent to the two other definitions of \lstinline|t_and_t|
\end{itemize}
\end{frame}

\begin{frame}[fragile]{Tactic: \lstinline|intros|}
\lstinputlisting{content/theorem-proving/coq/snippets/p_q_p.v}
\begin{itemize}
  \item By \lstinline|intros|, arguments are assumed
  \pause
  \item They are now available as \textbf{hypotheses} in the context $Γ$
  \pause
  \item Correspondence to proof terms:\\
  Introduces \lstinline|fun x y ... => ...|
\end{itemize}
\end{frame}

\begin{frame}[fragile]{Tactic: \lstinline|destruct| (1)}
\lstinputlisting{content/theorem-proving/coq/snippets/pq_p.v}
\begin{itemize}
  \item By \lstinline|destruct|, a hypothesis is case-analyzed
  \pause
  \item In this example, there is only one case, \lstinline|conj|
  \pause
  \item Correspondence to proof terms:\\
  Introduces \lstinline|match ... with conj p q => ...|
\end{itemize}
\end{frame}

\begin{frame}[fragile]{Tactics: \lstinline|destruct| (2), \lstinline|left|, \lstinline|right|}
\lstinputlisting{content/theorem-proving/coq/snippets/pq_or_qp.v}
\begin{itemize}
  \item By \lstinline|destruct|, a hypothesis is case-analyzed
  \pause
  \item For each case, there is a subgoal
  \pause
  \item Correspondence to proof terms:\\
  Introduces \lstinline&match ... with | ... | ... => ...&
  \item By \lstinline|left| (\lstinline|right|), the first (second) constructor is selected
  \item Correspondence to proof terms:\\
  Introduces \lstinline|or_introl| resp. \lstinline|or_intror|
\end{itemize}
\end{frame}

\begin{frame}[fragile]{Tactic: \lstinline|split|}
\lstinputlisting{content/theorem-proving/coq/snippets/and_comm.v}
\begin{itemize}
  \item By \lstinline|split|, a goal is split into subgoals
  \pause
  \item For each case, there is a subgoal
  \pause
  \item Correspondence to proof terms:\\
  Introduces \lstinline&conj ... ...&
\end{itemize}
\end{frame}

\begin{frame}[fragile]{Tactic: \lstinline|exfalso|}
\lstinputlisting{content/theorem-proving/coq/snippets/exfalso.v}
\begin{itemize}
  \item \lstinline|exfalso| replaces the current goal by \lstinline|False|
  \pause
  \item In other words, proving \lstinline|False| suffices to prove any \lstinline|P|
  \pause
  \item The correspondence to proof terms is very interesting. Recall that \lstinline|False| is an empty type. What happens if we assume a proof of \lstinline|False| (as in example)? As with every value, we can case-analyze it and prove \lstinline|P| \textbf{for every case}. But there are \textbf{no cases}, so we are done!
  \pause
  \item Crucial part of the corresponding proof term:\\
  \lstinline[mathescape=true]|match f with $(\mathit{nothing~here)}$ end|
\end{itemize}
\end{frame}

\begin{frame}[fragile]{Tactics: \lstinline|simpl|, \lstinline|reflexivity|}
\begin{lstlisting}
Lemma negb_tf: negb true = false.
Proof.
  simpl.
  reflexivity.
Qed.
\end{lstlisting}
\begin{itemize}
  \item By \lstinline|simpl|, a goal is maximally reduced
  \pause
  \item This yields the subgoal \lstinline|false =|~\lstinline|false|
  \pause
  \item By \lstinline|reflexivity|, we can prove such a goal
  \pause
  \item How is \lstinline|=| encoded as a type and what proof term does \lstinline|reflexivity| introduce?
  \pause
  \item The answer is \say{as an inductive type} but the details are not relevant at this point
\end{itemize}
\end{frame}

\begin{frame}[fragile]{Negation}
\begin{itemize}
  \item We now come to our fourth logical connective, negation

  \pause

  \item You have seen it, as it is already built-in: The negation of a proposition \lstinline|P| is the implication \lstinline|P -> False|

  \pause

  \item We use the notation \lstinline|~P| for \lstinline|P -> False|

  \pause

  \item Example: Prove that for all propositions \lstinline|P|, we have \lstinline|P -> ~(~P)|.
\lstinputlisting{content/theorem-proving/coq/snippets/not_not.v}
\end{itemize}
\end{frame}


% \begin{frame}[fragile]{Inductive Predicates, 1}
% \begin{itemize}
%   \item Consider the inductive predicate \lstinline|even|
% \lstinputlisting{content/theorem-proving/coq/snippets/even.v}
% \pause
%   \item Read as:
%   \begin{itemize}
%     \item \say{0 is even}
%     \item \say{if $n$ is even, so is $2+n$}
%     \item \say{nothing else is even} (essence of \textit{inductiveness})
%   \end{itemize}
%   \pause
%   \item The following type means \say{2 is even}
% \begin{lstlisting}
% even 2: Prop
% \end{lstlisting}
% \end{itemize}
% \end{frame}
%
% \begin{frame}[fragile]{Inductive Predicates, 2}
% \lstinputlisting{content/theorem-proving/coq/snippets/even.v}
% \begin{itemize}
%   \item \lstinline|even 2| is a proposition (a type that represents a logical statement)
%   \pause
%   \item This says nothing about whether it holds or not
%   \pause
%   \item We are sure though that \lstinline|even 2| should hold if we got the definition of \lstinline|even| right... let's prove it.
% \begin{lstlisting}
% Definition two_even: even 2 :=
%   evenSS (evenO).
% \end{lstlisting}
% \end{itemize}
% \end{frame}
%
% \begin{frame}[fragile]{Inductive Predicates, 3}
% \begin{itemize}
%   \item Here is another example of using tactics, this time to prove an inductive proposition
%   \pause
%   \item The following two definitions are equivalent
% \end{itemize}
% \begin{lstlisting}
% Definition two_even : even 2 :=
%   evenSS O evenO.
% \end{lstlisting}
% \begin{lstlisting}
% Lemma two_even : even 2.
% Proof.
%   apply evenSS. apply evenO.
% Qed.
% \end{lstlisting}
% \begin{itemize}
%   \pause
%   \item The tactics \lstinline|apply evenSS. apply evenO.| \textbf{generate} the term \lstinline|evenSS O evenO|
% \end{itemize}
% \end{frame}


\begin{frame}[fragile]{Types that Depend on Terms}
\begin{itemize}
  \item Recall polymorphic types, i.e. functions \textbf{from types}

  \begin{block}{3) Polymorphic Types}
    \begin{itemize}
      \item \lstinline|forall (X: Type), list X| \hfill \lstinline|: Type|
        \begin{itemize}
          \item with value \lstinline|nil|
        \end{itemize}
      \item \lstinline|forall (P: Prop), True \/|~\lstinline|P| \hfill \lstinline|: Prop|
        \begin{itemize}
          \item with proof \lstinline|fun (_: Prop) => or_introl I|
        \end{itemize}
    \end{itemize}
  \end{block}

  \pause

  \item Types can also be functions \textbf{from terms}

  \begin{block}{4) Dependent Types}
    \begin{itemize}
      \item \lstinline|forall (b: bool),|~~~\lstinline|negb (negb b) =|~\lstinline|b| \hfill \lstinline|: Prop|
      \item \lstinline|forall (n: nat),|~~~~~\lstinline|0 + n =|~\lstinline|n| \hfill \lstinline|: Prop|
      \item \lstinline|forall (n: nat),|~~~~~\lstinline|n + 0 =|~\lstinline|n| \hfill \lstinline|: Prop|
      \item \lstinline|forall (m n: nat),|\,~\lstinline|m + n =|~\lstinline|n + m| \hfill \lstinline|: Prop|
    \end{itemize}
  \end{block}

  \pause

  \item \textit{Roadmap}: We prove all of the above properties!
\end{itemize}
\end{frame}


% \begin{frame}[fragile]{Types that Depend on Terms}
% \lstinputlisting{content/theorem-proving/coq/snippets/even.v}
% \lstinputlisting{content/theorem-proving/coq/snippets/list.v}
% \begin{itemize}
%   \pause
%   \item Both \lstinline|even| and \lstinline|list| are type constructors
%   \begin{itemize}
%     \item They are both functions that map to types (\lstinline|Prop| is just a special universe)
%   \end{itemize}
%   \pause
%   \item But:
%   \begin{itemize}
%     \item \lstinline|list| is a function from \textbf{types} (of \lstinline|Type|)
%     \item \lstinline|even| is a function from \textbf{terms} (of \lstinline|nat|)
%   \end{itemize}
%   \pause
%   \item $\Rightarrow$ Type constructors can construct types out of \textbf{terms}
%   \pause
%   \item More generally: In Coq, \textbf{types are terms}!
% \end{itemize}
% \end{frame}

% \begin{frame}[fragile]{Dependent product ($Π$-)types}
% \lstinputlisting{content/theorem-proving/coq/snippets/even.v}
% \begin{itemize}
%   \item \lstinline|forall (n: nat), even n -> even (S(S n))|\\
%   is a \textbf{$Π$-type}
%   \pause
%   \item A \textbf{term} of this $Π$-type is a function that takes \textit{any} \lstinline|n: nat|, a proof of \lstinline|even n| and returns a proof of \lstinline|even (2|~\lstinline|+n)|
%   \pause
%   \item For intuition: Why is it called a dep. \textbf{product} type?
%   \pause
%   \item Because it can be seen as a (very large) cartesian product (if you think of types as sets)
%   \begin{align*}
%        &\text{\lstinline|forall (n: nat), even n -> even (S (S n))|}\\
%        =~~&  Π_{n: \mathit{nat}} ~(\mathit{even}~n → \mathit{even}~(2+n))\\
%        =~~& (\mathit{even}~0 → \mathit{even}~2) \times (\mathit{even}~2 → \mathit{even}~4) \times ...
%   \end{align*}
% \end{itemize}
% \end{frame}

\begin{frame}[fragile]{Type Universes}
\begin{itemize}
  \item The type of a type is called a \textit{type universe}: Either \lstinline|Type| or \lstinline|Prop|\footnote{This is a simplified view that is sufficient for now.}
\begin{lstlisting}
bool : Type   True                  : Prop
nat  : Type   10 = 4             : Prop
              forall (P: Prop), True \/ P : Prop
\end{lstlisting}
\pause
  \item A type of \lstinline|Type| (e.g. \lstinline|nat|) contains \textbf{data values}
\begin{lstlisting}
O     : nat        true  : bool
S O   : nat        false : bool
plus  : nat -> nat -> nat
\end{lstlisting}
\pause
  \item A type of \lstinline|Prop| (e.g. \lstinline|True|) contains \textbf{proofs}
\begin{lstlisting}
I     : True
(fun (_: Type) => or_introl I)
      : forall (P: Prop), True \/ P
\end{lstlisting}
\end{itemize}
\end{frame}

\begin{frame}[fragile]{Back to Booleans, 1}
\lstinputlisting{content/theorem-proving/coq/snippets/negb_param.v}
\begin{itemize}
  \item Let's prove \lstinline|negb (negb b) =|~\lstinline|b| for all \lstinline|b|
  \pause
  \item The term \lstinline|negb (neg b)| does not reduce
  \pause
  \item Why not? \lstinline|negb| performs pattern matching, but since we don't know anything about \lstinline|b| (it could be \textit{any bool}), we don't know which case will match
  \pause
  \item But there are only two possible values for \lstinline|b|. So let's do a case analysis and prove every case!
  \begin{enumerate}
    \item \lstinline|b| is \lstinline|true|. Then \lstinline|negb (negb true)| reduces to \lstinline|true|.
    \item \lstinline|b| is \lstinline|false|. Then \lstinline|negb (negb false)| reduces to \lstinline|false|.
  \end{enumerate}
\end{itemize}
\end{frame}

\begin{frame}[fragile]{Back to Booleans, 2}
\lstinputlisting{content/theorem-proving/coq/snippets/negb_param.v}
\begin{itemize}
  \item We have all the tools to prove this in Coq
\end{itemize}
  \pause
\begin{lstlisting}
Lemma negb_inverse:
  forall (b: bool), negb (negb b) = b.
Proof.
  intros b.
  destruct b.
  - simpl. reflexivity.
  - simpl. reflexivity.
Qed.
\end{lstlisting}

\end{frame}

\begin{frame}[fragile]{Back to Natural Numbers}
\lstinputlisting{content/theorem-proving/coq/snippets/plus_param.v}
\begin{itemize}
  \item We can easily prove that \lstinline|0 + n =| \lstinline|n| for all \lstinline|n|
  \pause
  \item Because: \lstinline|0 + n| reduces to \lstinline|n| by definition of \lstinline|plus|
  \pause
\end{itemize}
\lstinputlisting{content/theorem-proving/coq/snippets/zero_plus.v}
\end{frame}

\begin{frame}[fragile]{Natural induction}
\lstinputlisting{content/theorem-proving/coq/snippets/plus_param.v}
\begin{itemize}
  \item What about the other way round, \lstinline|m + 0 =|~\lstinline|m|?
  \pause
  \item This should hold, but we cannot reduce \lstinline|m +  0|
  \pause
  \item Why not? \lstinline|plus| pattern-matches on the first argument, but since we don't know anything about \lstinline|m| (it could be \textit{any number}), we don't know which case will match
  \pause
  \item \say{\textit{any number}} rings a bell: Proof by Natural Induction!
  \pause
  \item \textbf{Base case.} To show: $0 + 0 = 0$. By definition.
  \pause
  \item \textbf{Inductive case.} Let IH be $m + 0 = m$. To show: $(S~m) + 0 = S~m$. But this is by def. of \lstinline|plus| convertible to
  $S (m+ 0) = S~m$. Now use IH, so we have to show $S~m = S~m$ which holds by reflexivity of $=$.
\end{itemize}
\end{frame}

\begin{frame}[fragile]{Natural induction in Coq, 1}
\begin{itemize}
  \item Let's do the same proof, but in Coq
  \pause
\end{itemize}
\lstinputlisting{content/theorem-proving/coq/snippets/plus_zero.v}
\begin{itemize}
  \item \lstinline|induction n| does a case-analysis on \lstinline|n| (like \lstinline|destruct|), but provides an additional inductive hypothesis
  \pause
  \item \lstinline|rewrite IHn| uses the equation \lstinline|IHn| to substitute a subterm in the current goal
\end{itemize}
\end{frame}

\begin{frame}[fragile]{Natural induction in Coq, 2}
\begin{itemize}
  \item Remember the very first property we wanted to show?\\
  \lstinline|m + n =|~\lstinline|n + m| ~~for all \lstinline|m|, \lstinline|n|
  \pause

  \item Proof by induction over \lstinline|m|.
  \begin{itemize}
    \item \textbf{Base case}. To show: $0 + n = n + 0$. By our last lemma, we know $n + 0 = n$; by definition of \lstinline|plus| we know $0 + n = n$; thus we are done.
    \item \textbf{Inductive case}. Let IH be $m + n = n + m$. To show: $(S~m) + n = n + (S~m)$. This goal reduces to $S (m + n) = n + (S~m)$. By the IH, we can reduce the goal to $S (n + m) = n + (S~m)$. Here we have the same problem as with proving $n + 0 = n$: It does not hold by definition, because \lstinline|plus| pattern-matches on the \textit{first} argument. Thus we prove this as an extra lemma, after which the proof is completed.
  \end{itemize}
\end{itemize}
\end{frame}

\begin{frame}[fragile]{Natural induction in Coq, 3}
\begin{itemize}
  \item Let's prove the little lemma\\
  \lstinline|m + (S n) =|~\lstinline|S (m + n)| for all \lstinline|m|, \lstinline|n|\\
  ... by induction on \lstinline|m|
  \pause
\end{itemize}
\begin{lstlisting}
Lemma m_plus_S:
  forall (m n: nat), m +(S n) = S (m +n).
Proof.
  intros m n.
  induction m.
  - simpl. reflexivity.
  - simpl. rewrite IHm. reflexivity.
Qed.
\end{lstlisting}
\end{frame}

\begin{frame}[fragile]{Natural induction in Coq, 4}
\begin{itemize}
  \item Now we can prove commutativity of addition in Coq, following the proof sketch given before
  \pause
\end{itemize}
\begin{lstlisting}
Lemma plus_comm:
  forall (m n: nat), m +n = n +m.
Proof.
  intros m.
  induction m.
  - intros. rewrite m_plus_O.
    simpl. reflexivity.
  - intros n. simpl. rewrite IHm.
    rewrite m_plus_S. reflexivity.
Qed.
\end{lstlisting}
\end{frame}

\begin{frame}[fragile]{Induction in Coq: Outlook, 1}
\begin{itemize}
  \item We know the principle of natural induction from Maths
  \pause
  \item Why have we considered this principle a sound proof method?
  \pause
  \item Because our objects (here: natural numbers) are constructed out of \textbf{finitely} many steps\footnote{For further reading: \say{$x$ is-a-substructure-of $y$} is well-founded}. We can view a proof of induction as a recipe on how to obtain a proof for \textbf{any concrete} number $n$.
  \pause
  \item Example: How do we prove $3 + 0 = 3$?
    \begin{itemize}
      \item Use Inductive Case, but need to prove $2 + 0 = 2$. How?
      \item Use Inductive Case, but need to prove $1 + 0 = 1$. How?
      \item Use Inductive Case, but need to prove $0 + 0 = 0$. How?
      \item Use Base Case.
    \end{itemize}
  \pause
  \item Doesn't this proof construction look a lot like... a recursive program?
\end{itemize}
\end{frame}

\begin{frame}[fragile]{Induction in Coq: Outlook, 2}
\begin{itemize}
  \item Example: How do we prove $3 + 0 = 3$?
    \begin{itemize}
      \item Use Inductive Case, but need to prove $2 + 0 = 2$. How?
      \item Use Inductive Case, but need to prove $1 + 0 = 1$. How?
      \item Use Inductive Case, but need to prove $0 + 0 = 0$. How?
      \item Use Base Case.
    \end{itemize}
  \pause
  \item In Coq, an \textbf{inductive proof} is a \textbf{recursive function}
  \pause
  \item Every \lstinline|Inductive| type has this essential property that each object is constructed out of finitely many steps
  \pause
  \item There is an induction principle for every type, and it is automatically generated\footnote{You can take a look e.g. via \lstinline|Print nat_ind.|}
\end{itemize}
\end{frame}

\begin{frame}[fragile]{Summary}
  \begin{itemize}
    \item Coq is a \textbf{programming language} (with restricted recursion)
    \pause
    \item Data is defined inductively, i.e. all values are \textbf{finite objects}
    \pause
    \item Functions are defined recursively over this inductive structure
    \pause
    \item The Curry-Howard-Correspondence provides clever tricks to \textbf{encode propositions as types}
    \pause
    \item A proposition is proven by a well-typed \textbf{proof term}
    \pause
    \item A \textbf{proof by induction} is a recipe for constructing proofs for \textbf{any} element
      \begin{itemize}
        \item This recipe is a recursive function!
      \end{itemize}
    \pause
    \item Tactics assist the user in finding a proof term
  \end{itemize}
\end{frame}


% \begin{frame}[fragile]{Inductive Predicates, 3}
% \begin{itemize}
%   \item Compare the two types
% \end{itemize}
% \lstinputlisting{content/theorem-proving/coq/snippets/even.v}
% \lstinputlisting{content/theorem-proving/coq/snippets/list.v}
% \begin{itemize}
%   \item Remarkable difference 2:
%   \begin{itemize}
%     \item \lstinline|list| \say{lets us choose} \lstinline|X: Type|, we can build anything
%     \item \lstinline|even| allows only the construction for certain \lstinline|n: nat|
%   \end{itemize}
%   \item $\Rightarrow$ \textbf{Param.} (\lstinline|X| in \lstinline|list|) vs. \textbf{non-param.} (\lstinline|n| in \lstinline|even|)
% \end{itemize}
% \end{frame}
