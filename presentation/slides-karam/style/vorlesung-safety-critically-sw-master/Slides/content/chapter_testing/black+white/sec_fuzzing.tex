
\subsection{Fuzzing}


\begin{frame}{Problems}
	\begin{beameritemize}
		\item Programs that take structured inputs (Interfaces, Protocols) are often hard to test.
		
		\item To show robustness or security properties an unfeasible amount of manual testing would be necessary.
		
	\end{beameritemize}

\end{frame}


\begin{frame}{Fuzzing}
	Idea: Find errors by providing randomly generated data as input to a program.
	
	\begin{figure}
		\begin{tikzpicture}[scale=0.40]
		\node[state, initial, field] (fuzzer) {Fuzzer};
		\node[state, field, right=of fuzzer] (input) {Random Input};
		\node[state, field, right=of input] (SuT) {System Under Test};   
		\node[state, field, error , below =1cm of SuT] (error) {Error Found!};
		\node[state, field, success, below left=1cm and 0.5cm of SuT] (success) {No Error};
		\node[state, field, below=of error] (bug) {Bug Report}; 
		
		\draw[->] (success.west) -| (fuzzer.south);	
		
		\path[->]
		(fuzzer) edge node {} (input)
		(input) edge node {} (SuT)
		(SuT.south) edge node {} (error.north)
		(SuT.south) edge node {} (success.north)
		(error) edge node {} (bug)	
		;
		\end{tikzpicture}
	\end{figure}
	
\end{frame}



\begin{frame}{Fuzzing}
	Inputs can be generated:
	\begin{beameritemize}
		\item Completely random
		\item With awareness of the input structure
		\item With awareness of the program structure
	\end{beameritemize}

\end{frame}


\begin{frame}{Fuzzing}
	Issues that can be typically found during fuzzing:
	\begin{beameritemize}
		\item Invalid input formats
		\item Parsing issues
	\end{beameritemize}	

	\begin{block}{Effectiveness}
	The effectiveness of a fuzzer can be considered as the coverage it achieves.
\end{block}
\end{frame}